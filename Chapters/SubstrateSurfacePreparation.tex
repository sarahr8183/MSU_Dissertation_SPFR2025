\chapter{Substrate Surface Preparation}
\label{chSurfacePrep}

\section{Diamond Polishing}


\section{Hydrogen Plasma Etching}


\section{Reactive Ion Etching (RIE)}


\section{Effects of Surface Preparation on Deposition Quality}
	The lack of correlation in spectroscopic and morphological results with pressure and offcut variation and the weak correlation between substrate roughness with spectroscopic and morphological results presented the need for a study to understand the effects of substrate surface preparation on growth morphology and defect incorporation during deposition. A study to analyze these effects is in progress. One 3 mm x 3 mm x 0.5 mm (100) HPHT substrate from NDT has been fully characterized in the as-received condition from the supplier which included characterization of offcut, FTIR, UVVis, Raman, and AFM. The sample was then acid and solvent cleaned in the same way as the samples from the pressure optimization study. A deposition on the sample in the as-received condition was completed at 950 $^{\circ}$C, 250 mbar, and flow rates of 368 sccm of H$_2$, 16 sccm of CH$_4$, and 16.5 sccm of 0.01 mol\% N$_2$ in H$_2$, because these conditions gave the best result in the previous study. A 10-minute hydrogen etch was performed immediately prior to the deposition.
	
	The sample was fully characterized after deposition to understand the results for an as-received substrate. The as-received substrate produced a rough, dark deposition layer that appeared to be graphitic carbon or unepitaxial crystallites. 
	
	Next, the deposited layer will be mechanically polished back to the initial substrate surface. The substrate will then receive a fine mechanical polish to prepare the surface better than the as-received condition. The pre-characterization, deposition, post-characterization, and substrate surface preparation will be repeated several times only changing the substrate surface preparation technique. In addition to the fine mechanical polish, a chemo-mechanical polish, a significant hydrogen etch, and a reactive ion etch (RIE) will be compared for the preparation of the surface and analyze the effects of sample preparation on growth morphology and defect incorporation.