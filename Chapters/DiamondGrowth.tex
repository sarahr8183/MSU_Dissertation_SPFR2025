\chapter{Diamond Growth} 
\label{DiamondGrowth}
This chapter will discuss several different types of microwave plasma assisted chemical vapor deposition (CVD) diamond reactors. First, a brief introduction to microwave plasma reactors will be provided. Next, the bell jar style and NIRIM style reactor will be discussed. Finally, the diamond reactor used in this dissertation will be detailed.

\section{Microwave Plasma Assisted Chemical Vapor Depositions Diamond Reactors}
\indent
Microwave plasma reactors are a subset of chemical vapor deposition reactors. As discussed in chapter \ref{chIntro}, chemical vapor deposition synthetic diamond reactors succeeded high pressure high temperature (HPHT) diamond reactors. The HPHT technique was developed to establish the thermodynamically favorable conditions for diamond growth. Chemical vapor deposition reactor utilized reaction kinetics at the surface of a diamond seed to begin diamond growth. In chemical vapor deposition reactors, the vapor phase contains a carbon species, typically methane, which reacts with a solid surface, a diamond seed, to grow diamond one carbon-carbon bond at a time. Graphitic carbon-carbon bonds can also form during the CVD process. To mitigate this, a predominately hydrogen plasma is used to preferentially etch away the $sp^2$ bonded carbon, graphite.

The reaction between the vapor phase and solid surface needs energy to progress the reaction. This activation energy has been achieved using a heat source, such as in hot filament style reactors. Alternatively, a microwave supply can be used to generate a plasma and achieve the activation energy. Microwaves are coupled to the chamber which couples to gas phase creating radical reactive species.  Two variations of microwave plasma assisted CVD reactors are the bell jar style reactor and the NIRIM style reactor. These two reactor types will be discussed in greater detail in sections \ref{NIRIM} and \ref{ASTEX} respectively.


\subsection{NIRIM Style Reactor}
\label{NIRIM}
The first style of microwave plasma assisted reactor designed was the NIRIM reactor. A schematic of this reactor style is shown in Figure *make a figure and reference it*. Here, a quartz tube in inserted into a rectangular microwave waveguide and a laminar flow of process gases are flown over through the quartz tube. The microwaves couple to the gas phase within the quartz tube to create a plasma above the sample. This reactor style has significant size constraints, however the laminar flow of process gases is ideal for creating delta doped layered structures that is needed for ideal laser writing substrates. 

\subsection{ASTeX Bell Jar Style Reactor}
\label{ASTEX}
\indent
The initial bell jar style reactor was designed by Bachmann and team in the late 1980s \cite{Bachmann1991}. Later, Applied Science Technology Inc. (ASTEX) commercialized this reactor. This reactor creates a plasma inside a quartz dome directly above the sample. A schematic of this reactor style is shown in Figure *make a figure and reference it*. This quartz dome is encloses the vacuum chamber where the process gases are present. A microwave resonant cavity creates a standing wave where the node, encapsulate by the quartz bell jar, generates a sphereical plasma.

\subsection{MSU Style Diamond System 4}
\indent 
\label{DS4}
The MSU style diamond reactor is a bell jar style reactor that was patented in 1994 \cite{Asmussen1994}. A schematic for Diamond System 4 (DS4), an MSU style reactor, is shown in Figure \ref{fig:DS4Schematic}. The MSU style reactor generates a disk shaped plasma as opposed to the spherical plasma generated in the ASTEX bell jar style reactor. The disk shape plasma is generated by encapsulating one node of the standing wave in the quartz bell jar. 

\begin{figure}[h]
    \centering
    \includegraphics[width=0.5\linewidth]{figures/DS4Diagram.png}
    \caption{Schematic of Diamond System 4 (DS4) an MSU style bell jar diamond reactor.}
    \label{fig:DS4Schematic}
\end{figure}

